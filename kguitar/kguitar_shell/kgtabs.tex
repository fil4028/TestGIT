% FILE: kgtabs.tex
% This file contains some macro definitions for use with MusiXTeX.
% It was tested with MusiXTeX T.98, T.99.
%
% Most symbols will look ugly on your dvi-viewer, because there is
% some postscript code to put a white box behind them. The postscript
% file will look a lot better. Special thanks to Stanislav Kneifl
% for helping with this.
%
%%%%%%%%%%%%%%%%%%%%%%%%%%%%%%%%%%%%%%%%%%%%%%%%%%%%%%
% This file is based on tabdefs.tex Version 0.1 1998
% by Ronald Gelten, tgakrg@chem.tue.nl
%
% Special Thanks to R. Gelten who allowed us to make
% changes.
%%%%%%%%%%%%%%%%%%%%%%%%%%%%%%%%%%%%%%%%%%%%%%%%%%%%%%
%
% This file is required to use exported MusiXTeX tabs from KGuitar
%
% http://kguitar.sourceforge.net

\immediate\write10{KGuitarTABdefs 0.1.1\space<2000>}%
\immediate\write10{based on tabdefs.tex by R. Gelten}%

\chardef\tabnil=0
\chardef\tabone=1
\chardef\tabtwo=2
\chardef\tabthr=3
\chardef\tabfou=4
\chardef\tabfiv=5
\chardef\tabsix=6
\chardef\tabsev=7
\chardef\tabeig=8
\chardef\tabnin=9
\let\tabreverse\tabnil
\def\reversestrings{\let\tabreverse\tabone}
\def\guitarstrings{\let\tabreverse\tabnil}

\newcount\tabinstrument
\def\settab#1{%
  \tabinstrument=#1%
  \setlines#1 6%
  \setclefsymbol#1\tabclef%
}
%%%%%%%%%%%%%%%%%%%%%%%%%%%%%%
% define fonts for later use %
%%%%%%%%%%%%%%%%%%%%%%%%%%%%%%
\font\tabfntiii=cmss3
\font\tabfntiv=cmss4
\font\tabfntv=cmss5
\font\tabfntvi=cmss6
\font\tabfntvii=cmss7
\font\tabfntviii=cmss8
\font\tabfntix=cmss9
\font\tabfntx=cmss10
\font\tabfntxi=cmss11
\font\tabfntxii=cmss12
\font\tabfntxiv=cmss14
%
\font\tabbfntvi=cmssdc6
\font\tabbfntvii=cmssdc7
\font\tabbfntviii=cmssdc8
\font\tabbfntix=cmssdc9
\font\tabbfntx=cmssdc10
\font\tabbfntxi=cmssdc11
\font\tabbfntxii=cmssdc12
\font\tabbfntxiv=cmssdc14
%
\font\tablrfntviii=cmss8
\font\tablrfntx=cmss10
\font\tablrfntxi=cmss11
\font\tablrfntxii=cmss12
\font\tablrfntxiv=cmss14
%
\font\tabbbfntviii=cmssdc8
\font\tabbbfntx=cmssdc10
\font\tabbbfntxi=cmssdc11
\font\tabbbfntxii=cmssdc12
\font\tabbbfntxiv=cmssdc14
%
\font\disclaimerfont=cmsl8
\let\mymetertinyfont\eightbf
%%%%%%%%%%%%%%%%%%%%%%%%%%%%%%%%%%%%%%%%%%%%%%%%%%%%%%%%%%%%%%%%%%
% Define tabsizes. These set fonts and distances for staff lines %
% similar to \smallmusicsize etc. \largetabsize is default.      %
% Also set number of stafflines to 6.                            %
%%%%%%%%%%%%%%%%%%%%%%%%%%%%%%%%%%%%%%%%%%%%%%%%%%%%%%%%%%%%%%%%%%
\def\Largetabsize{%
  \setsize1{1.44} % \Largevalue
  \let\tabmeterfnt\meterlargefont
  \let\tabcleffnt\tabbfntxiv
  \let\tabfnt\tabfntx
  \let\tabbfnt\tabbfntx
  \let\tablrfnt\tablrfntx
  \let\tabbbfnt\tabbbfntx
  \let\chordfnt\tabfntxii
  \let\chordbfnt\tabbfntxii
  \let\stringfnt\tabfntix
}
\def\largetabsize{%
  \setsize1{1.2} % = \largevalue}
  \let\tabmeterfnt\meterbigfont
  \let\tabcleffnt\tabbfntxii
  \let\tabfnt\tabfntix
  \let\tabbfnt\tabbfntix
  \let\tablrfnt\tablrfntix
  \let\tabbbfnt\tabbbfntix
  \let\chordfnt\tabfntxi
  \let\chordbfnt\tabbfntxi
  \let\stringfnt\tabfntviii
}
\def\normaltabsize{%
  \setsize1{1} % = \normalvalue
  \let\tabmeterfnt\meternorfont
  \let\tabcleffnt\tabbfntx
  \let\tabfnt\tabfntvii
  \let\tabbfnt\tabbfntvii
  \let\tablrfnt\tablrfntvii
  \let\tabbbfnt\tabbbfntvii
  \let\tabbbfnt\tabbbfntvii
  \let\chordfnt\tabfntx
  \let\chordbfnt\tabbfntx
  \let\stringfnt\tabfntvi
}
\def\smalltabsize{%
  \setsize1{.8} % = \smallvalue
  \let\tabmeterfnt\metersmallfont
  \let\tabcleffnt\tabbfntviii
  \let\tabfnt\tabfntvi
  \let\tabbfnt\tabbfntvi
  \let\tablrfnt\tablrfntvi
  \let\tabbbfnt\tabbbfntvi
  \let\tabbbfnt\tabbbfntvi
  \let\chordfnt\tabfntviii
  \let\chordbfnt\tabbfntviii
  \let\stringfnt\tabfntv
}
\largetabsize
%%%%%%%%%%%%%%%%%%%%%%%%%%%%%%%%%%%%%%%%%%%%%%%%%%%%%%%%%%%%%%%%%
% \showstrings prints the string names at the left of the first %
% tabulature staff.                                             %
%%%%%%%%%%%%%%%%%%%%%%%%%%%%%%%%%%%%%%%%%%%%%%%%%%%%%%%%%%%%%%%%%
\def\showstrings#1#2#3#4#5#6{%
  \setbox0\hbox{\stringfnt{D}}%
  \setbox1\vbox{\baselineskip=1.5\internote%
    \hbox to 1.1\wd0{\hfil{\stringfnt{#1}}\hfil}
    \hbox to 1.1\wd0{\hfil{\stringfnt{#2}}\hfil}
    \hbox to 1.1\wd0{\hfil{\stringfnt{#3}}\hfil}
    \hbox to 1.1\wd0{\hfil{\stringfnt{#4}}\hfil}
    \hbox to 1.1\wd0{\hfil{\stringfnt{#5}}\hfil}
    \hbox to 1.1\wd0{\hfil{\stringfnt{#6}}\hfil}
  }%
  %\parindent=0pt
  \setname\tabinstrument{\box1}%
}
%%%%%%%%%%%%%%%%%%%%%%%%%%%%%%%%%%%%%%%%%%%%%%%%%%%%%%%%%%%%%%%%%%%%%
% Macros for tablature. They put numbers on the TAB-lines.          %
%%%%%%%%%%%%%%%%%%%%%%%%%%%%%%%%%%%%%%%%%%%%%%%%%%%%%%%%%%%%%%%%%%%%%
% \tabbox ensures a white background for the tab number in the ps-file
% (the stafflines don't overrun the number). Symbols are centered.
\def\tabbox#1#2{\setbox0=\hbox{\tabfnt #2}%
  \special{ps: 1 setgray}%
  \ccharnote{#1}{\vrule height \ht0 width \wd0 depth \dp0}%
  \special{ps: 0 setgray}%
  \ccharnote{#1}{\box0}%
}
% \ltabbox does the same as \tabbox, except that it produces 
% left-outlined symbols
\def\ltabbox#1#2{\setbox0=\hbox{\tabfnt #2}%
  \special{ps: 1 setgray}%
  \zcharnote{#1}{\vrule height \ht0 width \wd0 depth \dp0}%
  \special{ps: 0 setgray}%
  \zcharnote{#1}{\box0}%
}
% \rtabbox does the same as \tabbox, except that it produces 
% right-outlined symbols
\def\rtabbox#1#2{\setbox0=\hbox{\tabfnt #2}%
  \special{ps: 1 setgray}%
  \lcharnote{#1}{\vrule height \ht0 width \wd0 depth \dp0}%
  \special{ps: 0 setgray}%
  \lcharnote{#1}{\box0}%
}
% \tab takes two arguments. 
% #1 is the string number 
% #2 is the symbol to be put on the string
% the symbols are horizontally centered and a \sk is put behind it,
% to jump to the position of the next symbol
% expandafter<token>: TeX first reads the token that comes
% immediately after \expandafter, without expanding it. Let's call this
% token t. Then TeX reads the token that comes after t (and possibly
% more tokens, if that token has an argument), replacing it by its
% expansion. Finally TeX puts t back in front of that expansion.
\def\tab#1#2{%
  \ifcase\tabreverse
  \expandafter\tabbox{\ifcase#1\or n\or l\or j\or h\or f\or d\fi}{#2}\sk%
  \or
  \expandafter\tabbox{\ifcase#1\or d\or f\or h\or j\or l\or n\fi}{#2}\sk%
  \fi
}
% left spilling \tab
\def\ltab#1#2{%
  \ifcase\tabreverse
  \expandafter\ltabbox{\ifcase#1\or n\or l\or j\or h\or f\or d\fi}{#2}\sk%
  \or
  \expandafter\ltabbox{\ifcase#1\or d\or f\or h\or j\or l\or n\fi}{#2}\sk%
  \fi
}
% right spilling  \tab
\def\rtab#1#2{%
  \ifcase\tabreverse
  \expandafter\rtabbox{\ifcase#1\or n\or l\or j\or h\or f\or d\fi}{#2}\sk%
  \or
  \expandafter\rtabbox{\ifcase#1\or d\or f\or h\or j\or l\or n\fi}{#2}\sk%
  \fi
}
% \chord-tab symbol. Same as \tab, but no \sk is given, so multiple 
% symbols can be placed above one another
\def\chotab#1#2{%
  \ifcase\tabreverse
  \expandafter\tabbox{\ifcase#1\or n\or l\or j\or h\or f\or d\fi}{#2}%
  \or
  \expandafter\tabbox{\ifcase#1\or d\or f\or h\or j\or l\or n\fi}{#2}%
  \fi
}
% left spilling \chotab
\def\lchotab#1#2{%
  \ifcase\tabreverse
  \expandafter\ltabbox{\ifcase#1\or n\or l\or j\or h\or f\or d\fi}{#2}%
  \or
  \expandafter\ltabbox{\ifcase#1\or d\or f\or h\or j\or l\or n\fi}{#2}%
  \fi
}
% right spilling \chotab
\def\rchotab#1#2{%
  \ifcase\tabreverse
  \expandafter\rtabbox{\ifcase#1\or n\or l\or j\or h\or f\or d\fi}{#2}%
  \or
  \expandafter\rtabbox{\ifcase#1\or d\or f\or h\or j\or l\or n\fi}{#2}%
  \fi
}
%
%%%%%%%%%%%%%%%%%%%%%%%%%%%%%%%%%%%%%%%%%%%%
% define macros to put chords above lyrics %
%%%%%%%%%%%%%%%%%%%%%%%%%%%%%%%%%%%%%%%%%%%%
\def\minvskip{\vskip -0.2\baselineskip}
\def\plsvskip{\vskip 0.2\baselineskip}
\def\chord#1{\tabfnt #1}
%
\def\above#1#2#3#4{\setbox0=\hbox{#1}\setbox1=\hbox{#2}%
\ifnum\wd0>\wd1\vbox{\plsvskip \hbox to \wd0{#3#2#4} \minvskip \hbox{#1}}%
\else\vbox{\plsvskip \hbox{#2} \minvskip \hbox to \wd1{#3#1#4}}\fi}
\def\cabove#1#2{\above{#1}{\tabfnt #2}{\hfil}{\hfil}}
\def\labove#1#2{\above{#1}{\tabfnt #2}{}{\hfil}}
\def\rabove#1#2{\above{#1}{\tabfnt #2}{\hfil}{}}
%
%%%%%%%%%%%%%%%%%%%%%%%%%%%%%%%%%%%%%%%%%%%%%%%%%%
% define a new clefsymbol \tabclef for tablature %
%%%%%%%%%%%%%%%%%%%%%%%%%%%%%%%%%%%%%%%%%%%%%%%%%%
\def\tabclef{%
  \setbox0=\hbox{\tabcleffnt T}
  \setbox1=\vbox{\baselineskip=0.5\baselineskip
    \hbox to 1.1\wd0{{\tabcleffnt T}\hfil}
    \hbox to 1.1\wd0{{\tabcleffnt A}\hfil}
    \hbox to 1.1\wd0{{\tabcleffnt B}\hfil}
  }
  %\special{ps: 1 setgray}%
  %\zcharnote{d}{\vrule height 0.9\ht1 width \wd0 depth \dp0}%
  %\special{ps: 0 setgray}%
  \lower1.5\internote\box1}
%%%%%%%%%%%%%%%%%%%%%%%
% Macros to set meter %
%%%%%%%%%%%%%%%%%%%%%%%
\makeatletter
\def\meterfrac#1#2{%
  \ifnum\noinstrum@nt=\tabinstrument% adapted copy from musixtex.tex
    \setbox\toks@box\vbox{\hbox{~\tabmeterfnt#1}%
    \hbox{~\tabmeterfnt #2}}\raise\internote%
    \vbox\@to\@ight\internote{\offinterlineskip%
    \vss\hbox\@to\wd\toks@box{\hss\tabmeterfnt#1\hss}\vskip1.5\internote%
    \vss\hbox\@to\wd\toks@box{\hss\tabmeterfnt#2\hss}\vss}%
  \else% copy from musixtex.tex
    \setbox\toks@box\vbox{\hbox{~\meterfont#1}%
    \hbox{~\meterfont #2}}%
    \vbox\@to\@ight\internote{\offinterlineskip%
    \vss\hbox\@to\wd\toks@box{\hss\meterfont#1\hss}\vss%
    \vss\hbox\@to\wd\toks@box{\hss\meterfont#2\hss}\vss}%
  \fi%
}
\def\allabreve{% -> 2/2
  \ifnum\noinstrum@nt=\tabinstrument\raise5\internote\hbox{\keychar82}%
  \else\raise4\internote\hbox{\keychar82}\fi}
\def\reverseallabreve{%
  \ifnum\noinstrum@nt=\tabinstrument\raise5\internote\hbox{\keychar123}%
  \else\raise5\internote\hbox{\keychar123}\fi}
\def\meterC{% -> 4/4
  \ifnum\noinstrum@nt=\tabinstrument%
  \raise5\internote\hbox{\musickeyfont\@lxxxiii}%
  \else\raise4\internote\hbox{\musickeyfont\@lxxxiii}\fi}
\def\reverseC{% -> 4/2
  \ifnum\noinstrum@nt=\tabinstrument\raise5\internote\hbox{\keychar84}%
  \else\raise4\internote\hbox{\keychar84}\fi}
\makeatother
%%%%%%%%%%%
% repeats %
%%%%%%%%%%%
% source copied from musixtex.tex
% writ@colons is redefined so that it still uses \tabcoli for the
% tabulature-instrument, but keeps using \w@coli for all other instruments.
\makeatletter
\def\writ@colons{\n@loop{\ifnum\noinstrum@nt=\tabinstrument\tabcoli%
                         \else\w@coli\fi}\repeat%
}
% \tabcoli places two dots, it is a slightly adapted version of \w@coli
\newskip\tabcolskip
\tabcolskip1.2\internote
\def\tabcoli{\p@loop\raise\altportee%
  \hbox{%
    \raise5\internote\rlap{\kernm\tabcolskip\hbox{\keychar24}}%
    \raise1\internote\rlap{\kernm\tabcolskip\hbox{\keychar24}}%\kernm\tabcolskip
  }\count@portee\repeat%
}
\makeatother
%%%%%%%%%%%%%%%%%%
% set disclaimer %
%%%%%%%%%%%%%%%%%%
\def\disclaimerbox#1#2#3{\setbox0=\hbox{#1}\setbox1=\hbox{#2}%
\ifnum\wd1>\wd0\vbox{\hbox to \wd1{#3#1} \vskip -.5\baselineskip \hbox{#2}}%
\else\vbox{\hbox{#1} \vskip -.5\baselineskip \hbox to \wd0{#3#2}}\fi}
%
\def\disclaimer{\shipout\vbox{\vbox to \vsize{%
\vss\pagecontents\vss}\line{%
\hfill\disclaimerbox%
{\disclaimerfont{This transcription is the author's own interpretation.}}%
{\disclaimerfont{It may only be used for private study, scholarship, %
                or research.}}%
\hfil%
}}}
%
\def\disclineone{This transcription is the author's own interpretation.}
\def\disclinetwo{It may only be used for private study, scholarship, or
                 research.}
\def\putdisclaimer{\advance\vsize-24pt\footline{%
\hfill\disclaimerbox%
{\disclaimerfont{\disclineone}}%
{\disclaimerfont{\disclinetwo}}%
\hfill
}}
%%%%%%%%%%%%%%%%%%%%%%%%%%%%%%%%%%%%%%%%%%
% some symbols defined for personal use. %
%%%%%%%%%%%%%%%%%%%%%%%%%%%%%%%%%%%%%%%%%%
\def\x{$\times$}
\def\forte{\zcharnote r{\f}}
\def\ppiano{\zcharnote r{\pp}}
\def\n{\hfil\break}
\def\slidesymbol{\,\slide{e}{1}{5}~}
\def\noslide{\setbox0=\hbox{$\slidesymbol$ }\hbox to \wd0{~}}
\def\narslide{\slide{e}{1}{5}}
\def\arpsymbol{\,\,\,\arpeggio d 2}
\def\tuning#1#2{\hbox to 1.5cm{\circleit{#1}\quad = #2\hfil}}
\def\legend#1#2{\setbox0\hbox{\circleit{1}\quad = }%
    \hbox{\hbox to \wd0{\hfil #1 =} \hbox{#2}}}
\def\bprime{{\chordbfnt{'}}}
\def\chord#1{{\chordfnt{#1}}}
\def\chordb#1{{\chordbfnt{#1}}}
\def\upchord#1{\zcharnote q{\chordfnt{#1}}}
